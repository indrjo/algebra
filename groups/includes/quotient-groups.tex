% !TEX program = lualatex
% !TEX spellcheck = en_GB
% !TEX root = ../groups.tex

\section{Quotient groups}

Consider a group \(G\) and an equivalence relation \(\sim\) on it: we have the quotient set \(G/\sim\). Is it a group? Not always, but we really do want to have \q{quotient groups}. We stick to the case where \(\sim\) is compatible with the operation with the operation on a group, that is
\[a \sim b \text{ and } c \sim d \rarr ac \sim bd \quad \text{for every } a, b, c, d \in G.\]
Above all, such \(G/\sim\) must have a magmatic structure, that is having a well-defined operation
\begin{equation}(G/\sim) \times (G/\sim) \to G/\sim\,, \ (\bar x, \bar y) \to \bar x \ast \bar y \coloneq \bar{xy}.\label{eqn:QuotOp}
\end{equation}
The compatibility of \(\sim\) fits the tasks. To appreciate this, imagine \(\sim\) is not compatible. There exists \(a, b, c, d \in G\) such that \(a \sim b\), \(c \sim d\) and not \(ac \sim bd\). In this case we would have \(\bar a \ast \bar c = \bar b \ast \bar d\) but \(\bar{ac} \ne \bar{bd}\).\newline
We overcame the initial hurdle, because the group structure naturally follows without any other nuisance:

\begin{proposition}\label{prop:QuotientGroupEquivRel}
If \(G\) is a group and \(\sim\) is an equivalence relation on \(G\) compatible with its operation, then \(G/\sim\) with the operation~\eqref{eqn:QuotOp} is a group.
\end{proposition}

\begin{proof}
Straightforward and quite boring\dots{} daily routine.
\end{proof}

The relations \(\call_H\) and \(\calr_H\) have a particular role in Algebra.

\begin{proposition}\label{prop:LRCompatible}
Let \(G\) be a group and \(H\) a subgroup of \(G\). Then \(\call_H\) is compatible with the operation of \(G\) if and only if 
\begin{equation}
xh\inv x \in H \text{ for every } x \in G, h \in H .\label{eqn:NormalSubGrp}
\end{equation}
The same holds for \(\calr_H\).
\end{proposition}

\begin{proof}
Obviously, \(x \call_H xh\) for every \(x \in G\) and \(h \in G\). If \(\call_H\) is compatible with the operation \(G\) comes with, then \(x \inv x \call_H x h \inv x\), that is \(1 \call_H x h \inv x\). In this case, \(x h \inv x = k\) for some \(k \in H\), so \(xh\inv x \in H\).\newline
Assume now~\eqref{eqn:NormalSubGrp}. Consider \(a, b, c, d \in G\) such that \(a \call_H b\) and \(c \call_H d\). We have \(ahck=bd\) for some \(h, k \in H\). But \(\inv chc \in H\), that is \(hc = c h'\) for some \(h' \in H\); thus \(bd = (ac)(h'k)\), viz \(ac \call_H bd\), and we have finished.
\end{proof}

So the subgroups \(H\) satisfies~\eqref{eqn:NormalSubGrp} have a special role: they are the ones and the only ones such that \(G/\call_H\) and \(G/\calr_H\) have a group structure in the sense we have explained above. Such subgroups deserve a special name.

\begin{definition}[Normal subgroups]
For \(G\) group, a subgroup \(H\) of \(G\) is said {\em normal} whenever \(xh\inv x \in H\) for every \(x \in G\) and \(h \in H\).
\end{definition}

However, more is true:

\begin{proposition}
Let \(G\) be a group and \(H\) a subgroup of \(H\). Then the following facts are equivalent:
\begin{enumerate}
\item \(H\) is normal;
\item \(xH = Hx\) for every \(x \in G\);
%\item \(xH \subseteq Hx\) for every \(x \in G\);
\item \(xH\inv x = H\) for every \(x \in G\).
\end{enumerate}
\end{proposition}

\begin{proof}
Left as exercise, but quite simple.
\end{proof}

\begin{corollary}
Let \(G\) be a group and \(H\) a finite subgroup of \(G\). If \(H\) is the unique subgroup of \(G\) that has cardinality \(n\), then it is \(H\) is normal.
\end{corollary}

\begin{proof}
If \(H\) is a subgroup of \(G\), so is \(x H \inv x\) for each \(x \in G\). Besides, both have the same cardinality, hence \(H = xH\inv x\).
\end{proof}

\begin{corollary}
Let \(G\) be a group and \(H\) a subgroup of \(G\). If \([G:H] = 2\), then \(H\) is normal.
\end{corollary}

\begin{proof}
One element of \(G/\call_H\) is \(H\) itself and, since \(G/\call_H\) is a partition of \(G\), the other one is \(G \setminus H\); the same occurs in \(G/\calr_H\). Hence \(xH = H = Hx\) if \(x \in H\), otherwise \(xH = G \setminus H = Hx\). We can conclude \(H\) is normal.
\end{proof}

\begin{definition}
For \(G\) group and \(H\) a normal subgroup of \(G\), the group
\[G/H \coloneq G/\call_H = G/\calr_H = \set{xH \mid x \in G}\]
is the {\em quotient group} of \(G\) through \(H\). It is a group in the sense the set that \(G/H\) has the operation
\begin{align*}
G/H \times G/H & \to G/H \\
(xH, yH) & \to (xH)(yH) \coloneq (xy)H
\end{align*}
(this operation is well-defined by Proposition~\ref{prop:QuotientGroupEquivRel} and Proposition~\ref{prop:LRCompatible}) \(H\) is the identity and \(\inv{(xH)} = \inv x H\) for every \(x \in G\).
\end{definition}


